\begin{listing}[H]
    	\caption{Horizontal and vertical kernels}
    	\label{listing:code1}
    	\begin{minted}[frame=lines, framesep=2mm,baselinestretch=1.1,fontsize=\footnotesize,linenos]{cpp}
#include <iostream>
#include <cmath>
#include <opencv2/opencv.hpp>

using namespace cv;

Mat img, imgGray;

void doImageProcessing() {
	
	//The image is really high resolution for making many windows, so I use 
	//the opencv function resize, to make it more manageable
	resize(img, img, Size(640, 480));
	
	imshow("original image", img);
	
	//Converting image to grayscale
	cvtColor(img, imgGray, CV_BGR2GRAY);
	imshow("Original gray scale image", imgGray);
	
	//Standard Sobel kernel
	int kernelX[3][3] = {1, 0, -1, 2, 0, -2, 1, 0, -1};
	
	
	//Standard Sobel kernel
	int kernelY[3][3] = {1, 2, 1, 0, 0, 0, -1, -2, -1};
	
	
	int radius = 1;
	
	Mat src = imgGray.clone();
	
	//Saving the current "initial" image
	Mat gradX = imgGray.clone();
	Mat gradY = imgGray.clone();
	Mat gradF = imgGray.clone();
	
	//Looping over the the image with the x kernel
	//From this we get the gradient image
	for (int row = radius; row < src.rows - radius; row++) {
		for (int col = radius; col < src.cols - radius; col++) {
			int scale = 0;
			for (int i = -radius; i <= radius; i++) {
				for (int j = -radius; j <= radius; j++) {
					scale += 
					src.at<uchar>(row + i, col + j) * kernelX[i + radius][j + radius];
				}
			}
			gradX.at<uchar>(row - radius, col - radius) = scale / 60;
		}
	}
    	\end{minted}
    \end{listing}


\begin{listing}[H]
	\caption{Horizontal and vertical kernels}
	\label{listing:code2}
	\begin{minted}[frame=lines, framesep=2mm,baselinestretch=1.1,fontsize=\footnotesize,linenos]{cpp}
imshow("X edge detection", gradX);
//Looping over the image with the y kernel
//From this we get the gradient image
for (int row = radius; row < src.rows - radius; row++) {
	for (int col = radius; col < src.cols - radius; col++) {
		int scale = 0;
		
		for (int i = -radius; i <= radius; i++) {
			for (int j = -radius; j <= radius; j++) {
				scale += 
				src.at<uchar>(row + i, col + j)* kernelY[i + radius][j + radius];
			}
		}
		gradY.at<uchar>(row - radius, col - radius) = scale / 60;
	}
}

imshow("Y edge detection", gradY);


//Here we calculate an approximation of the gradient at every point, using both the x and y images
for (int row = 0; row < gradF.rows; row++) {
	for (int col = 0; col < gradF.cols; col++) {
		
		gradF.at<uchar>(row, col) = static_cast<uchar>(sqrt(pow(gradX.at<uchar>(row, col), 2)
		 + pow(gradY.at<uchar>(row, col), 2)));
		//If the magnitude of the resulting pixel is higher than 240, max it
		//Else zero it, thus making the image binary, and kewl
		if (gradF.at<uchar>(row, col) > 1) {
			gradF.at<uchar>(row, col) = 255;
		} else {
			gradF.at<uchar>(row, col) = 0;
		}
	}
}

imshow("Edges", gradF);

waitKey(0);
}

int main() {

img = imread("/home/daniel/Documents/opencvFilters/stars.jpeg", CV_LOAD_IMAGE_UNCHANGED);

if (img.empty()) return -1;

doImageProcessing();
return 0;
}
\end{minted}
\end{listing}